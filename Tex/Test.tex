\documentclass{beamer}

\usepackage[UTF8, scheme = plain]{ctex}
\usetheme{Warsaw}
%\usecolortheme{beaver}

\usepackage{graphicx}

\usepackage{listings}
\lstset{language = C++, numberstyle=\tiny,keywordstyle=\color{blue!70},commentstyle=\color{red!50!green!50!blue!50},rulesepcolor=\color{red!20!green!20!blue!20},basicstyle=\ttfamily\small}


\title{动态规划}
\author{陶鸿杰}
\institute{QQ:1170755856}
\date{2019/05/05}

%-------------------------------------------------------------------------------------------------
\begin{document}

%---------------------------section A-----------------------------------------
\section{简单的问题}
\subsection{介绍}
\frame{\titlepage}

%---------------------------------------------
\subsection{数字三角形}
\begin{frame}
\frametitle{\href{http://poj.org/problem?id=1163}{数字三角形(POJ1163)}}

%------------左右分栏------
\begin{minipage}[b]{0.30\linewidth}
	\includegraphics[scale=0.5]{pic/NumTriangle.jpg}
\end{minipage}
\hfill
\begin{minipage}[b]{0.55\linewidth}

	 在左侧的数字三角形中寻找一条从顶部到底边的路径,使得路径上所经过的数字之和最大。\\ 
	 
	    \begin{itemize}
	 	\item 路径上的每一步都只能往左下或 右下走。
	 	\item 只需要求出这个最大和即可,不必给出具体路径。
	 	\item 三角形的行数大于1,小于等于100,数字为 0 - 99。
	 \end{itemize}
\end{minipage}
\end{frame}

\begin{frame}[fragile]
\frametitle{问题分析:}

%------------左右分栏------
\begin{minipage}[b]{0.30\linewidth}
	\includegraphics[scale=0.5]{pic/NumTriangle.jpg}
\end{minipage}
\hfill
\begin{minipage}[b]{0.55\linewidth}
	\begin{itemize}
		\item 三角形的行数大于1,小于等于100,数字为 0 - 99。
		\item 暴力的时间复杂度是 $O(2^n)$,而 $2^{100}$ 肯定会超时
		\item 在不考虑记忆话搜索的情况下,我们需要怎么解决这个问题呢?
		
	\end{itemize}
	
\end{minipage}
\end{frame}



\begin{frame}[fragile]
\frametitle{问题分析:}

%------------左右分栏------
\begin{minipage}[b]{0.30\linewidth}
	\includegraphics[scale=0.5]{pic/NumTriangle.jpg}
\end{minipage}
\hfill
\begin{minipage}[b]{0.55\linewidth}
	\iffalse
	三角形的行数最大是100,暴力的时间复杂度是 $O(2^n)$,而 $2^{100}$ 肯定会超时\\
	在不考虑记忆话搜索的情况下,我们需要怎么解决这个问题呢?\\
	我们以a[i][j]表示这个数字三角形第i行j列上的数字
	\fi
	\begin{itemize}
		\item 用a[i][j]存储这个数字三角形第i行的第j个数字
		\item 当前的位置dp[i][j]看成一个状态
		\item 则dp[i][j]的状态只能由它的上一层头顶上的那两个状态转移到的
	\end{itemize}
	
\end{minipage}
\end{frame}

\begin{frame}[fragile]
\frametitle{问题分析:}

%------------左右分栏------
\begin{minipage}[b]{0.30\linewidth}
	\includegraphics[scale=0.5]{pic/NumTriangle.jpg}
\end{minipage}
\hfill
\begin{minipage}[b]{0.55\linewidth}
\begin{itemize}
	\item 我们已经知道对于任意一个位置,它的状态dp[i][j]是由它的上面两个状态转移过来的\\
	\item 我们反过来想,如果我们让路径从下往上走,那么:对于任意一个位置,它的状态dp[i][j]是由它的下面两个状态转移过来的
\end{itemize}

\end{minipage} 
\begin{block}{状态转移方程:}
  $dp[i][j] = a[i][j] + max(dp[i+1][j], dp[i+1][j+1])$
\end{block}

\end{frame}

\begin{frame}[fragile]
\frametitle{我们可以写出代码了}
\begin{block}{核心代码  (复杂度为$O(n^2)$)}
	\begin{lstlisting}
for(int i=n-1;i>0;i--)
    for(int j=1;j<=i;j++)
	a[i][j] += max(a[i+1][j],a[i+1][j+1]);
printf("%d\n",a[1][1]);
	\end{lstlisting}
\end{block}
动态规划
\end{frame}
%--------------------------------动态规划解题的一般思路------------------------
\begin{frame}[fragile]
\frametitle{动态规划解题的一般思路}

\begin{minipage}[b]{0.30\linewidth}
	\begin{figure}
		\centering
		\includegraphics[scale=0.40]{pic/NumTriangle.jpg}
	\end{figure}
	
\end{minipage}
\hfill
\begin{minipage}[b]{0.65\linewidth}
	
	\begin{block}{1.将原问题分解为子问题}
		\begin{itemize}
			\item 把原问题分解为若干个子问题,子问题和原问题形式相同或类似,只不过规模变小了。子问题都解决,原问题即解决。
			\item 子问题的解一旦求出就会被保存,所以每个子问题只需求解一次。
		\end{itemize}
	\end{block}
\end{minipage}



\end{frame}

\begin{frame}[fragile]
\frametitle{动态规划解题的一般思路}

\begin{minipage}[b]{0.30\linewidth}
\begin{figure}
	\centering
	\includegraphics[scale=0.40]{pic/NumTriangle.jpg}
\end{figure}

\end{minipage}
\hfill
\begin{minipage}[b]{0.65\linewidth}

\begin{block}{2.确定状态}
	\begin{itemize}
		\item 动态规划解题时,我们往往将和子问题相关的各个变量的一组取值,称之为一个“状态”。
		\item 一个“状态”对应于一个或多个子问题,某个“状态”下的“值”,就是这个“状态”所对应的子问题的解。
	\end{itemize}
\end{block}
\end{minipage}
\end{frame}

\begin{frame}[fragile]
\frametitle{动态规划解题的一般思路}

\begin{minipage}[b]{0.30\linewidth}
\begin{figure}
\centering
\includegraphics[scale=0.40]{pic/NumTriangle.jpg}
\end{figure}

\end{minipage}
\hfill
\begin{minipage}[b]{0.65\linewidth}

\begin{block}{3.确定一些初始状态(边界状态)的值}
\begin{itemize}
	\item 就以“数字三角形”为例,初始状态就是底边数字,值就是底边数字值。
\end{itemize}
\end{block}
\end{minipage}
\end{frame}

\begin{frame}[fragile]
\frametitle{动态规划解题的一般思路}

\begin{minipage}[b]{0.30\linewidth}
	\begin{figure}
		\centering
		\includegraphics[scale=0.40]{pic/NumTriangle.jpg}
	\end{figure}
	
\end{minipage}
\hfill
\begin{minipage}[b]{0.65\linewidth}
	
	\begin{block}{4.确定状态转移方程}
		\begin{itemize}
			\item 定义出什么是“状态”,以及在该“状态”下的“值”后,就要找出不同的状态之间如何迁移——即如何从一个或多个“值”已知的“状态”,求出另一个“状态”的“值”(“人人为我’递推型)。状态的迁移可以用递推公式表示,此递推公式也可被称作“状态转移方程”。
		\end{itemize}
	\end{block}
\end{minipage}
\end{frame}

\begin{frame}[fragile]
\frametitle{能用动规解决的问题的特点}
\begin{block}{问题具有最优子结构的性质}
	\begin{itemize}
		\item 如果问题的最优解所包含的子问题的解也是最优的,我们就称该问题具有最优子结构性质
	\end{itemize}
\end{block}
\begin{block}{无后效性}
	\begin{itemize}
		\item 当前的若干个状态值一旦确定,则此后过程的演变就只和这若干个状态的值有关,和之前是采取哪种手段或经过哪条路径演变到当前的这若干个状态,没有关系。
	\end{itemize}
\end{block}
\end{frame}

%---------------------------------------------
\subsection{矩阵取数}
\begin{frame}
%\frametitle{\href{https://www.51nod.com/Challenge/Problem.html\!\problemId=1083}{矩阵取数问题(51Nod 1083)}}
\frametitle{矩阵取数(51Nod1083)}

%------------左右分栏------
\begin{minipage}[b]{0.30\linewidth}
	\includegraphics[scale=0.7]{pic/num2.jpg}
\end{minipage}
\hfill
\begin{minipage}[b]{0.55\linewidth}
	
给定一个m行n列的矩阵,矩阵每个元素是一个正整数,你现在在左上角(第一行第一列),你需要走到右下角(第m行,第n列),每次只能朝右或者下走到相邻的位置,不能走出矩阵。走过的数的总和作为你的得分,求最大的得分。

\end{minipage}
\end{frame}

\begin{frame}[fragile]
%\frametitle{\href{https://www.51nod.com/Challenge/Problem.html\!\problemId=1083}{矩阵取数问题(51Nod 1083)}}
\frametitle{问题分析}

%------------左右分栏------
\begin{minipage}[b]{0.30\linewidth}
	\includegraphics[scale=0.7]{pic/num23.jpg}
\end{minipage}
\hfill
\begin{minipage}[b]{0.55\linewidth}
贪心可以么?	\\
我们先把大的数字走完!~ \\
但是....\\
无论我们以什么方式走到3,总和都是$1 + 1 + 3 + 1 + 1 + 1 + 1 = 9$ \\
	
\end{minipage}


\end{frame}

\begin{frame}[fragile]
%\frametitle{\href{https://www.51nod.com/Challenge/Problem.html\!\problemId=1083}{矩阵取数问题(51Nod 1083)}}
\frametitle{问题分析}

%------------左右分栏------
\begin{minipage}[b]{0.30\linewidth}
	\includegraphics[scale=0.7]{pic/num24.jpg}
\end{minipage}
\hfill
\begin{minipage}[b]{0.55\linewidth}
我们为了1个3,放弃了那么多个2, 不值啊\\
如果我们放弃3而走那些2, 得到的和是$1 + 1 + 2 + 2 + 2 + 1 + 1 = 10$ \\	
	
\end{minipage}

看来贪心是错的! 因为我们走到最大值时有很多值不能走到了,一个最大值会把我们带沟里去
\end{frame}


\begin{frame}[fragile]
%\frametitle{\href{https://www.51nod.com/Challenge/Problem.html\!\problemId=1083}{矩阵取数问题(51Nod 1083)}}
\frametitle{问题分析}

%------------左右分栏------
\begin{minipage}[b]{0.30\linewidth}
	\includegraphics[scale=0.8]{pic/num25.jpg}
\end{minipage}
\hfill
\begin{minipage}[b]{0.55\linewidth}
	
	
\end{minipage}

你觉得枚举可行么?\\
你需要往下走(m – 1)次,往右走(n – 1)次,才能走到右下角,一共走m + n – 2步!可行路径的总条数有$C_{m + n – 2}^{m - 1}$。
\end{frame}

\begin{frame}[fragile]
%\frametitle{\href{https://www.51nod.com/Challenge/Problem.html\!\problemId=1083}{矩阵取数问题(51Nod 1083)}}
\frametitle{问题分析}

%------------左右分栏------
\begin{minipage}[b]{0.30\linewidth}
	\includegraphics[scale=0.8]{pic/num25.jpg}
\end{minipage}
\hfill
\begin{minipage}[b]{0.55\linewidth}
	
	
\end{minipage}

当矩阵的行数和列数都等于100的时候,来让我们计算一下:\\
$C_{1198}^{99}=22750883079422934966181954039568885395604168260154104734000$
\end{frame}

\begin{frame}[fragile]
%\frametitle{\href{https://www.51nod.com/Challenge/Problem.html\!\problemId=1083}{矩阵取数问题(51Nod 1083)}}
\frametitle{dp一下}

%------------左右分栏------
\begin{minipage}[b]{0.30\linewidth}
	\includegraphics[scale=0.7]{pic/num2.jpg}
\end{minipage}
\hfill
\begin{minipage}[b]{0.55\linewidth}
	\begin{itemize}
		\item 用a[i][j]存储这个二维矩阵的第i行第j个的数字
		\item 当前的位置dp[i][j]看成一个状态
		\item 则dp[i][j]的状态只能由它的头顶上的状态和它左侧的那个状态转移到的
	\end{itemize}
	
\end{minipage}
\end{frame}

\begin{frame}[fragile]
%\frametitle{\href{https://www.51nod.com/Challenge/Problem.html\!\problemId=1083}{矩阵取数问题(51Nod 1083)}}
\frametitle{dp一下 }
\begin{minipage}[b]{0.30\linewidth}
	\includegraphics[scale=0.7]{pic/num2.jpg}
\end{minipage}
\hfill
\begin{minipage}[b]{0.55\linewidth}
	那么我们就可以写出来状态转移方程了 \\
	状态转移方程如下所示:\\
\end{minipage}
\begin{block}{状态转移方程}
\includegraphics[scale=0.7]{pic/dp1.jpg}
\end{block}
\end{frame}



\begin{frame}[fragile]
\frametitle{核心代码}
\begin{block}{代码实现  (复杂度为$O(n^2)$)}
\begin{lstlisting}
for(int i=1;i<=n;i++)
  for(int j=1;j<=n;j++)
    if(i==1)    
	a[i][j] += a[i][j-1];
    else if(j==1)
	a[i][j] += a[i-1][j];
    else
	a[i][j] += max(a[i-1][j],a[i][j-1]);
printf("%d\n",a[n][n]);
\end{lstlisting}
\end{block}
\end{frame}


\begin{frame}[fragile]
%\frametitle{\href{https://www.51nod.com/Challenge/Problem.html\!\problemId=1083}{矩阵取数问题(51Nod 1083)}}
\frametitle{状态分析}
\begin{minipage}[b]{0.30\linewidth}
	\includegraphics[scale=0.7]{pic/num26.jpg}
\end{minipage}
\hfill
\begin{minipage}[b]{0.55\linewidth}
	\begin{block}{代码分析(i=1时)}
	对于第j列,它的状态只能由它左侧的状态来传递的\\
	即 $dp[i][j] = a[i][j] + dp[i][j-1]$\\
	箭头的方向即为状态转移的方向
	\end{block}
\end{minipage}
\end{frame}
\begin{frame}[fragile]
%\frametitle{\href{https://www.51nod.com/Challenge/Problem.html\!\problemId=1083}{矩阵取数问题(51Nod 1083)}}
\frametitle{状态分析}
\begin{minipage}[b]{0.30\linewidth}
	\includegraphics[scale=0.7]{pic/num27.jpg}
\end{minipage}
\hfill
\begin{minipage}[b]{0.55\linewidth}
	\begin{block}{代码分析(i=2时)}
	对于第j列,它的状态由它左侧的状态和它上面的状态来传递的\\
	即 $dp[i][j] = a[i][j] + max(dp[i][j-1], dp[i-1][j])$\\
	箭头的方向即为状态转移的方向
	\end{block}	
\end{minipage}
\end{frame}
\begin{frame}[fragile]
%\frametitle{\href{https://www.51nod.com/Challenge/Problem.html\!\problemId=1083}{矩阵取数问题(51Nod 1083)}}
\frametitle{状态分析}
\begin{minipage}[b]{0.30\linewidth}
	\includegraphics[scale=0.7]{pic/num28.jpg}
\end{minipage}
\hfill
\begin{minipage}[b]{0.55\linewidth}
	\begin{block}{代码分析(i=3时)}
	对于第j列,它的状态由它左侧的状态和它上面的状态来传递的\\
	即 $dp[i][j] = a[i][j] + max(dp[i][j-1], dp[i-1][j])$\\
	箭头的方向即为状态转移的方向
	\end{block}	
\end{minipage}
\end{frame}



%---------------------------section B-----------------------------------------
\section{经典的问题}

%---------------------------------------------
\subsection{最大子段和}
\begin{frame}
%\frametitle{\href{https://www.51nod.com/Challenge/Problem.html#!#problemId=1049}{最大子段和(51Nod1049)}}
\frametitle{最大子段和(51Nod1049)}
%------------左右分栏------
\begin{block}{题目描述}
给出一个整数数组a(正负数都有),如何找出一个连续子数组(可以一个都不取,那么结果为0),使得其中的和最大?\\
例如:-2,11,-4,13,-5,-2,和最大的子段为:11,-4,13。和为20。
\end{block}
\begin{block}{输入}
	第1行:整数序列的长度N(2$ <= N <= 50000$)\\
	第2 - N + 1行:N个整数(-$10^9 <= A[i] <= 10^9$)
\end{block}	
\begin{block}{输出}
	输出最大子段和。
\end{block}

\end{frame}

\begin{frame}[fragile]
\frametitle{问题分析}
\begin{block}{例如}
	-2,11,-4,13,-5,-2\\
	和最大的子段为:11,-4,13。和为20。 
\end{block}
	枚举? \\
	让数组$a_i$保存第i个数字,那么,我们枚举的话,会得到这样的结果:\\
	\begin{itemize} \scriptsize
		\item \{$a_1$\}, \{$a_1+a_2$\}, \{$a_1+a_2+a_3$\}, \{$a_1+a_2+a_3+a_4$\}, \{$a_1+a_2+a_3+a_4+a_5$\} 
		\item \{$ a_2$\}, \{$ a_2+a_3$\}, \{$a_2+a_3+a_4$\}, \{$a_2+a_3+a_4+a_5$\} 
		\item \{$ a_3$\}, \{$ a_3+a_4$\}, \{$ a_3+a_4+a_5$\} 
		\item \{$a_4$\}, \{$a_4+a_5$\}
		\item \{$a_5$\} 
	\end{itemize}
\end{frame}

\begin{frame}[fragile]
\frametitle{问题分析}
\begin{block}{例如}
	-2,11,-4,13,-5,-2\\
	和最大的子段为:11,-4,13。和为20。 
\end{block}
\begin{itemize} \scriptsize
	\item \{$a_1$\}, \{$a_1+a_2$\}, \{$a_1+a_2+a_3$\}, \{$a_1+a_2+a_3+a_4$\}, \{$a_1+a_2+a_3+a_4+a_5$\} 
	\item \{$ a_2$\}, \{$ a_2+a_3$\}, \{$a_2+a_3+a_4$\}, \{$a_2+a_3+a_4+a_5$\} 
	\item \{$ a_3$\}, \{$ a_3+a_4$\}, \{$ a_3+a_4+a_5$\} 
	\item \{$a_4$\}, \{$a_4+a_5$\}
	\item \{$a_5$\} 
\end{itemize}
我们需要枚举起点,终点,以及计算它们的和,时间复杂度是$O(n^3)$\\
即使我们利用前缀和的方法,也只能把时间复杂度降低到$O(n^2)$,还是比较慢
\end{frame}
	
\begin{frame}[fragile]
\frametitle{问题分析}
\begin{block}{例如}
	-2,11,-4,13,-5,-2\\
	和最大的子段为:11,-4,13。和为20。 
\end{block}
\begin{itemize} \scriptsize
	\item \{$a_1$\}, \{$a_1+a_2$\}, \{$a_1+a_2+a_3$\}, \{$a_1+a_2+a_3+a_4$\}, \{$a_1+a_2+a_3+a_4+a_5$\} 
	\item \{$ a_2$\}, \{$ a_2+a_3$\}, \{$a_2+a_3+a_4$\}, \{$a_2+a_3+a_4+a_5$\} 
	\item \{$ a_3$\}, \{$ a_3+a_4$\}, \{$ a_3+a_4+a_5$\} 
	\item \{$a_4$\}, \{$a_4+a_5$\}
	\item \{$a_5$\} 
\end{itemize}
能不能在$O(n)$的时间内把这个问题解决?\\

\end{frame}


\begin{frame}[fragile]
\frametitle{问题分析}
\begin{block}{例如}
	-2,11,-4,13,-5,-2\\
	和最大的子段为:11,-4,13。和为20。 
\end{block}
\begin{itemize} \scriptsize
	\item \{$a_1$\}, \{$a_1+a_2$\}, \{$a_1+a_2+a_3$\}, \{$a_1+a_2+a_3+a_4$\}, \{$a_1+a_2+a_3+a_4+a_5$\} 
	\item \{$ a_2$\}, \{$ a_2+a_3$\}, \{$a_2+a_3+a_4$\}, \{$a_2+a_3+a_4+a_5$\} 
	\item \{$ a_3$\}, \{$ a_3+a_4$\}, \{$ a_3+a_4+a_5$\} 
	\item \{$a_4$\}, \{$a_4+a_5$\}
	\item \{$a_5$\} 
\end{itemize}
用动态规划的思路想一下:\\
如果我们用dp[i]记录第i个位置上的状态\\
让dp[i]表示以a[i]为结尾的最大子段和
\end{frame}

\begin{frame}[fragile]
\frametitle{问题分析}
\begin{block}{例如}
	-2,11,-4,13,-5,-2\\
	和最大的子段为:11,-4,13。和为20。 
\end{block}
\begin{itemize} \scriptsize
	\item \{$a_1$\}, \{$a_1+a_2$\}, \{$a_1+a_2+a_3$\}, \{$a_1+a_2+a_3+a_4$\}, \{$a_1+a_2+a_3+a_4+a_5$\} 
	\item \{$ a_2$\}, \{$ a_2+a_3$\}, \{$a_2+a_3+a_4$\}, \{$a_2+a_3+a_4+a_5$\} 
	\item \{$ a_3$\}, \{$ a_3+a_4$\}, \{$ a_3+a_4+a_5$\} 
	\item \{$a_4$\}, \{$a_4+a_5$\}
	\item \{$a_5$\} 
\end{itemize}

那么对于第i个数字a[i], 我们可以取, 也可以不取\\
如果加上a[i]后, $dp[i]<0$, 那么就不能取a[i],这个时候dp[i]=0 \\
否则dp[i]就等于dp[i-1]+a[i],这个就是以a[i]为结尾的最大子段和
\end{frame}


\begin{frame}[fragile]
\frametitle{问题分析}
\begin{block}{例如}
	-2,11,-4,13,-5,-2\\
	和最大的子段为:11,-4,13。和为20。
\end{block}
\begin{itemize}
	\item dp数组初始化为0
	\item dp[i] = max(dp[i-1]+a[i], dp[i])
	\item 遍历一遍dp数组,取最大的
\end{itemize}
	
\end{frame}

\begin{frame}[fragile]
\frametitle{DP代码}

\begin{block}{代码实现  (复杂度为$O(n)$)}
	\begin{lstlisting}
memset(dp,0,sizeof(dp));
ll maxx = 0;
for(int i=1;i<=n;i++)
{
    cin>>a[i];
    dp[i] = max(dp[i-1]+a[i],a[i]);
    maxx = max(maxx,dp[i]);
}
cout<<maxx<<endl;
	\end{lstlisting}
\end{block}
\end{frame}


%---------------------------------------------

\subsection{最长公共子序列}

\begin{frame}
\frametitle{\href{http://poj.org/problem?id=1458}{最长公共子序列(POJ 1458)}}
%\frametitle{最长公共子序列(POJ 1458)}
\begin{block}{最长公共子序列定义}
	最长公共子串(Longest Common Substring)和最长公共子序列(Longest Common Subsequence, LCS)的区别:\\
	子串(Substring)是串的一个连续的部分,子序列(Subsequence)则是从不改变序列的顺序,而从序列中去掉任意的元素而获得的新序列;\\
	更简略地说,前者(子串)的字符的位置必须连续,	后者(子序列LCS)则不必。\\
	比如字符串acdfg同akdfc的最长公共子串为df,而他们的最长公共子序列是adf。LCS可以使用动态规划法解决。
\end{block}	

\end{frame}

\begin{frame}[fragile]
%\frametitle{\href{https://www.51nod.com/Challenge/Problem.html\!\problemId=1083}{矩阵取数问题(51Nod1083)}}
%\frametitle{最长公共子序列(51Nod1006)}
\frametitle{\href{http://poj.org/problem?id=1458}{最长公共子序列(POJ 1458)}}
\begin{block}{题目描述}
	给出两个字符串A B,求A与B的最长公共子序列的长度。\\
	比如两个串为:\\
	\begin{itemize}
		\item abcfbc         
		\item abfcab
	\end{itemize}
	ab是两个串的子序列,abc也是,abfc也是,其中abfc是这两个字符串最长的子序列。\\
	所以结果是4
\end{block}
\end{frame}

\begin{frame}[fragile]
%\frametitle{\href{https://www.51nod.com/Challenge/Problem.html\!\problemId=1083}{矩阵取数问题(51Nod1083)}}
%\frametitle{最长公共子序列(51Nod1006)}
\frametitle{\href{http://poj.org/problem?id=1458}{最长公共子序列(POJ 1458)}}
\begin{block}{DP一下}
	不妨定义$dp[i][j]$表示$A_1\cdots A_i$和$B_1\cdots B_j$对应的LCS的长度\\
	那么, $A_1\cdots A_{i+1}$和$B_1\cdots B_{j+1}$对应的公共子序列可能是\\
	\begin{itemize}\footnotesize	
		\item 当$A_{i+1}=B_{j+1}$时, 在$A_1\cdots A_{i}$和$B_1\cdots B_{j}$的公共子序列末尾追加上$A_{i+1}$
		\item  $A_1\cdots A_{i}$和$B_1\cdots B_{j+1}$的公共子序列
		\item  $A_1\cdots A_{i+1}$和$B_1\cdots B_{j}$的公共子序列
	\end{itemize}
	三者中的某一个,所以就有了一下递推关系成立
	$$dp[i+1][j+1] =
	\begin{cases}
	dp[i][j]+1 & A_{i+1} =B_{j+1} \\
	max(dp[i+1][j],dp[i][j+1]) & else
	\end{cases}$$
\end{block}
\end{frame}

\begin{frame}[fragile]
\frametitle{DP代码}
\begin{block}{代码实现  (复杂度为$O(n*m)$)}
	\begin{lstlisting}
for(int i=0;i<n;++i)
  for(int j=0;j<m;++j)
    if(a[i]==b[j])  
      dp[i+1][j+1] = dp[i][j] + 1;
    else            
      dp[i+1][j+1] = max(dp[i+1][j], dp[i][j+1]);
cout<<dp[n][m]<<endl;
	\end{lstlisting}
\end{block}
\end{frame}



%-----------------------------下面是注释掉的代码-------------------------------------------
\iffalse
\subsection{最长公共子序列}
\begin{frame}
\frametitle{\href{http://poj.org/problem?id=1458}{最长公共子序列(POJ 1458)}}
%\frametitle{最长公共子序列(POJ 1458)}

%------------左右分栏------
\begin{minipage}[b]{0.30\linewidth}
\includegraphics[scale=0.7]{pic/num2.jpg}
\end{minipage}
\hfill
\begin{minipage}[b]{0.55\linewidth}

给定一个m行n列的矩阵,矩阵每个元素是一个正整数,你现在在左上角(第一行第一列),你需要走到右下角(第m行,第n列),每次只能朝右或者下走到相邻的位置,不能走出矩阵。走过的数的总和作为你的得分,求最大的得分。

\end{minipage}
\end{frame}

\begin{frame}[fragile]
\frametitle{DP代码}
\begin{block}{代码实现  (复杂度为$O(n^2)$)}
\begin{lstlisting}
for(int i=1;i<=n;i++)
for(int j=1;j<=n;j++)
if(i==1)    
a[i][j] += a[i][j-1];
else if(j==1)
a[i][j] += a[i-1][j];
else
a[i][j] += max(a[i-1][j],a[i][j-1]);
printf("%d\n",a[n][n]);
\end{lstlisting}
\end{block}
\end{frame}
\fi
%-----------------------------注释部分结束-------------------------------------------


%---------------------------------------------
\subsection{最长上升子序列}
\begin{frame}
%\frametitle{\href{https://www.51nod.com/Challenge/Problem.html\!\problemId=1083}{矩阵取数问题(51Nod1083)}}
%\frametitle{\href{http://poj.org/problem?id=1631}{最长上升子序列(POJ1631)}}
\frametitle{最长上升子序列}
\begin{block}{最长上升子序列问题}
	有一个长为$\ n\ $的数列$\;a_0, a_1, \dots, a_{n-1}$。请求出这个序列中最长的上升子序列的长度。上升子序列指的是对于任意的$i<j\ $都满足$\ a_i<a_j\ $的子序列。\\
	
	\begin{itemize}
		\item $1\le n \le 1000$
		\item $0\le a_i \le 1000000$
	\end{itemize}
\end{block}
\begin{block}{输入}
	5\\
	4 2 3 1 5
\end{block}
\begin{block}{输出}
	3 $\quad (a_1, a_2, a_4\ $构成的子序列$2, 3, 5$最长)
\end{block}
\end{frame}

\begin{frame}[fragile]
%\frametitle{\href{https://www.51nod.com/Challenge/Problem.html\!\problemId=1083}{矩阵取数问题(51Nod1083)}}
%\frametitle{\href{http://poj.org/problem?id=1631}{最长上升子序列(POJ1631)}}
\frametitle{最长上升子序列}
这一问题能够通过使用DP在$O(n^2)$的时间内解决,我们来建立一下递推关系。
\begin{itemize}
	\item 定义dp[i]:以$a_i$为末尾的最长上升子序列的长度
\end{itemize}
以$a_i$为结尾的上升子序列是
\begin{itemize}
	\item 只包含$a_i$的子序列
	\item 在满足$j<i$并且$a_j<a_i$的以$a_j$为结尾的上升子序列末尾,追加上$a_i$后得到的子序列
\end{itemize}
这二者之一。这样我们就到到了以下的递推关系:\\
$$dp[i] = max\{1, dp[j]+1|j<i \text{且} a_j<a_i\}$$
\end{frame}

\begin{frame}[fragile]
\frametitle{代码实现}
\begin{block}{DP代码}
	\begin{lstlisting}
int res = 0;
for(int i=0;i<n;++i)
{
  dp[i] = 1;
  for(int j=0;j<i;++j)
    if(a[j]<a[i])     
      dp[i] = max(dp[i], dp[j]+1);
  res = max(res, dp[i]);
}
cout<<res<<endl;
	\end{lstlisting}
\end{block}
\end{frame}

\begin{frame}[fragile]
%\frametitle{\href{https://www.51nod.com/Challenge/Problem.html\!\problemId=1083}{矩阵取数问题(51Nod1083)}}
%\frametitle{\href{http://poj.org/problem?id=1631}{最长上升子序列(POJ1631)}}
\frametitle{结束}
	学习算法需要多练多思考,这里给大家推荐几个学习动态规划地方(点击即可转到):\\
	\begin{itemize}
		\item \href{https://www.icourse163.org/learn/PKU-1001894005?tid=1205957211#/learn/content}{PKU郭炜老师讲的算法课程}
		\item \href{https://www.51nod.com/Tutorial/TutorialList.html}{51Nod里面的教程}
		\item \href{http://www.cppblog.com/menjitianya/archive/2015/10/23/212084.html}{夜深人静写算法(二) - 动态规划}
	\end{itemize}
\end{frame}


%---------多行注释-----|||||||||||||||||||||||||||||||||||||||||||||||||||||||||||||||
\iffalse
\begin{frame}[fragile]
\frametitle{Code:}
a[i][j]代表第i行j列的数字
\begin{block}{代码实现  (复杂度为$O(n^2)$)}
\begin{lstlisting}
for(int i=1;i<=n;i++)
for(int j=1;j<=n;j++)
if(i==1)    
a[i][j] += a[i][j-1];
else if(j==1)
a[i][j] += a[i-1][j];
else
a[i][j] += max(a[i-1][j],a[i][j-1]);
printf("%d\n",a[n][n]);
\end{lstlisting}
\end{block}
\end{frame}
\textbf{}


\begin{frame}[fragile]
\frametitle{算法描述}
数学表述:
\begin{itemize}
	\item ${\displaystyle \gcd(a,0)=a}$
	\item ${\displaystyle \gcd(a,b)=\gcd(b,a\,\mathrm {mod} \,b)} $
\end{itemize}
\begin{block}{代码实现  (复杂度为O(logn))}
	\begin{lstlisting}
	int gcd(a, b) {
	return b ? gcd(b, a % b) : a;
	}
	\end{lstlisting}
\end{block}
\end{frame}


\begin{frame}[fragile]
\frametitle{算法描述}
数学表述:
\begin{itemize}
	\item ${\displaystyle \gcd(a,0)=a}$
	\item ${\displaystyle \gcd(a,b)=\gcd(b,a\,\mathrm {mod} \,b)} $
\end{itemize}
\begin{block}{代码实现  (复杂度为O(logn))}
	\begin{lstlisting}
	int gcd(a, b) {
	return b ? gcd(b, a % b) : a;
	}
	\end{lstlisting}
\end{block}
\end{frame}


%--------------------------------------背包 问题---------------------------------------------------

\section{背包$\quad$问题}

%---------------------------------------------
\subsection{0$\;$1$\;$背包}
\begin{frame}
%\frametitle{\href{https://www.51nod.com/Challenge/Problem.html#!#problemId=1049}{最大子段和(51Nod1049)}}
\frametitle{最大子段和(51Nod1049)}
%------------左右分栏------
\begin{block}{题目描述}
	给出一个整数数组a(正负数都有),如何找出一个连续子数组(可以一个都不取,那么结果为0),使得其中的和最大?\\
	例如:-2,11,-4,13,-5,-2,和最大的子段为:11,-4,13。和为20。
\end{block}
\begin{block}{输入}
	第1行:整数序列的长度N(2$ <= N <= 50000$)\\
	第2 - N + 1行:N个整数(-$10^9 <= A[i] <= 10^9$)
\end{block}	
\begin{block}{输出}
	输出最大子段和。
\end{block}

\end{frame}

\begin{frame}[fragile]
\frametitle{Code:}
a[i][j]代表第i行j列的数字
\begin{block}{代码实现  (复杂度为$O(n^2)$)}
\begin{lstlisting}
for(int i=n-1;i>0;i--)
for(int j=1;j<=i;j++)
a[i][j] += max(a[i+1][j],a[i+1][j+1]);
printf("%d\n",a[1][1]);
\end{lstlisting}
\end{block}
\end{frame}

%---------------------------------------------
\subsection{完全背包}
\begin{frame}
%\frametitle{\href{https://www.51nod.com/Challenge/Problem.html\!\problemId=1083}{矩阵取数问题(51Nod1083)}}
\frametitle{最长公共子序列()}

%------------左右分栏------
\begin{minipage}[b]{0.30\linewidth}
\includegraphics[scale=0.7]{pic/num2.jpg}
\end{minipage}
\hfill
\begin{minipage}[b]{0.55\linewidth}

给定一个m行n列的矩阵,矩阵每个元素是一个正整数,你现在在左上角(第一行第一列),你需要走到右下角(第m行,第n列),每次只能朝右或者下走到相邻的位置,不能走出矩阵。走过的数的总和作为你的得分,求最大的得分。

\end{minipage}
\end{frame}

\begin{frame}[fragile]
\frametitle{Code:}
a[i][j]代表第i行j列的数字
\begin{block}{代码实现  (复杂度为$O(n^2)$)}
\begin{lstlisting}
for(int i=1;i<=n;i++)
for(int j=1;j<=n;j++)
if(i==1)    
a[i][j] += a[i][j-1];
else if(j==1)
a[i][j] += a[i-1][j];
else
a[i][j] += max(a[i-1][j],a[i][j-1]);
printf("%d\n",a[n][n]);
\end{lstlisting}
\end{block}
\end{frame}

%---------------------------------------------
\subsection{多重背包}
\begin{frame}
%\frametitle{\href{https://www.51nod.com/Challenge/Problem.html\!\problemId=1083}{矩阵取数问题(51Nod1083)}}
\frametitle{最长上升子序列()}

%------------左右分栏------
\begin{minipage}[b]{0.30\linewidth}
\includegraphics[scale=0.7]{pic/num2.jpg}
\end{minipage}
\hfill
\begin{minipage}[b]{0.55\linewidth}

给定一个m行n列的矩阵,矩阵每个元素是一个正整数,你现在在左上角(第一行第一列),你需要走到右下角(第m行,第n列),每次只能朝右或者下走到相邻的位置,不能走出矩阵。走过的数的总和作为你的得分,求最大的得分。

\end{minipage}
\end{frame}

\begin{frame}[fragile]
\frametitle{Code:}
a[i][j]代表第i行j列的数字
\begin{block}{代码实现  (复杂度为$O(n^2)$)}
\begin{lstlisting}
for(int i=1;i<=n;i++)
for(int j=1;j<=n;j++)
if(i==1)    
a[i][j] += a[i][j-1];
else if(j==1)
a[i][j] += a[i-1][j];
else
a[i][j] += max(a[i-1][j],a[i][j-1]);
printf("%d\n",a[n][n]);
\end{lstlisting}
\end{block}
\end{frame}
\textbf{}

%---------多行注释-----|||||||||||||||||||||||||||||||||||||||||||||||||||||||||||||||
%\iffalse
\begin{frame}[fragile]
\frametitle{算法描述}
数学表述:
\begin{itemize}
\item ${\displaystyle \gcd(a,0)=a}$
\item ${\displaystyle \gcd(a,b)=\gcd(b,a\,\mathrm {mod} \,b)} $
\end{itemize}
\begin{block}{代码实现  (复杂度为O(logn))}
\begin{lstlisting}
int gcd(a, b) {
return b ? gcd(b, a % b) : a;
}
\end{lstlisting}
\end{block}
\end{frame}


\begin{frame}[fragile]
\frametitle{算法描述}
数学表述:
\begin{itemize}
\item ${\displaystyle \gcd(a,0)=a}$
\item ${\displaystyle \gcd(a,b)=\gcd(b,a\,\mathrm {mod} \,b)} $
\end{itemize}
\begin{block}{代码实现  (复杂度为O(logn))}
\begin{lstlisting}
int gcd(a, b) {
return b ? gcd(b, a % b) : a;
}
\end{lstlisting}
\end{block}
\end{frame}



%-------------------------------------------------------------------------------------------------

\section{A}
\subsection{定义}
\begin{frame}
\frametitle{定义}
 如果用暴力匹配的思路,并假设现在文本串S匹配到 i 位置,模式串P匹配到 j 位置,则有:
\begin{itemize}
\item 如果当前字符匹配成功(即S[i] == P[j]),则i++,j++,继续匹配下一个字符;
\item 如果失配(即S[i]! = P[j]),令i = i - (j - 1),j = 0。相当于每次匹配失败时,i 回溯,j 被置为0

\end{itemize}
\begin{block}
	
\end{block}
\end{frame}

\subsection{辗转相除法}
\begin{frame}[fragile]
\frametitle{算法描述}
数学表述:
\begin{itemize}
\item ${\displaystyle \gcd(a,0)=a}$
\item ${\displaystyle \gcd(a,b)=\gcd(b,a\,\mathrm {mod} \,b)} $
\end{itemize}
\begin{block}{代码实现  (复杂度为O(logn))}
\begin{lstlisting}
int gcd(a, b) {
    return b ? gcd(b, a % b) : a;
}
\end{lstlisting}
\end{block}
\end{frame}

%-----------------------------------------------------------------------------------------------------
\section{B}
\subsection{定义}
\begin{frame}
\frametitle{定义}
\begin{itemize}
\item 若有一个数X,可以被另外两个数A、B整除,且X大于(或等于)A和B,则X为A和B的公倍数。A和B的公倍数有无限个,而所有的公倍数中,最小的公倍数就叫做最小公倍数。
\item 与最大公因数之关系 $$ \operatorname {lcm}(a,b)={\frac  {|a\cdot b|}{\operatorname {gcd}(a,b)}} $$
\end{itemize}
\end{frame}

%-----------------------------------------------------------------------------------------------------
\section{C}
\subsection{素数定义}
\begin{frame}
\frametitle{定义}
\begin{itemize}
\item 质数(Prime number),又称素数,指在大于1的自然数中,除了1和该数自身外,无法被其他自然数整除的数(也可定义为只有1与该数本身两个正因数的数)。大于1的自然数若不是素数,则称之为合数。
\item 例如,5是个素数,因为其正约数只有1与5。而6则是个合数,因为除了1与6外,2与3也是其正约数。
\end{itemize}
\end{frame}

\subsection{常见题型}
\begin{frame}
\frametitle{问题}
\begin{itemize}
\item $ n<10^{16} $,单次询问,判断n是否为素数
\item  $n<10^7 $,$ k<10^5 $,判断 $ a_1,a_2, ... , a_k < n $,是否为素数
\item  $n<10^{8} $,单次询问,求不超过 $n$ 的素数数量
\item  $l < n < r < ^{14} , r - l < 10^4 $,单次询问,求在区间 $[l, r]$ 中的素数数量
\end{itemize}
\end{frame}

\subsection{埃拉托斯特尼筛法}
\begin{frame}
\frametitle{埃拉托斯特尼筛法}
\begin{itemize}
\item 基本思想:素数的倍数一定不是素数
\item 实现方法:用一个长度为N+1的数组保存信息(0表示素数,1表示非素数),先假设所有的数都是素数(初始化为0),从第一个素数2开始,把2的倍数都标记为非素数(置为1),一直到大于N;然后进行下一趟,找到2后面的下一个素数3,进行同样的处理,直到最后,数组中依然为0的数即为素数。
\item 说明:整数1特殊处理即可。
\end{itemize}
\end{frame}

\begin{frame}[fragile]
\frametitle{比埃拉托斯特尼筛法}
prim[] 中值为0代表素数、1代表合数
\begin{block}{代码实现  (复杂度为O(nloglogn))}
\begin{lstlisting}
    prim[1] = 1;
    for(int i=2;i*i<TOP;i++)
        if(!prim[i]) for(int j=i*i;j<TOP;j+=i)
                prim[j]=1;
\end{lstlisting}
\end{block}
\end{frame}

\subsection{欧拉筛法}
\begin{frame}[fragile]
\frametitle{比埃拉托斯特尼筛法稍快的欧拉筛法}
掌握埃拉托斯特尼筛法即可,速度差不太远
\begin{block}{代码实现  (复杂度为O(n))}
\begin{lstlisting}
    for(int i = 2; i < MAXN; ++i) {
        if(!p[i])  prm[sz++] = i;
        for(int j = 0; j < sz; ++j) {
            int t = i * prm[j];
            if(t >= MAXN) break;
            p[t] = true;
            if(i%prm[j] == 0) break;
        }
    }
\end{lstlisting}
\end{block}
\end{frame}

%-----------------------------------------------------------------------------------------------------
\section{D}
\subsection{问题与应用}
\begin{frame}
\begin{itemize}
\item $a,n < INF, m < 10^{18}$ 求 $a^n \mod m$
\item $a_0, q < INF, m < 10^{18}$ 求 $a_n=a_0*q^n \mod m$
\end{itemize}
\end{frame}

\subsection{代码实现}
\begin{frame}[fragile]
\frametitle{代码实现}
\begin{block}{代码实现  (复杂度为O(logn))}
\begin{lstlisting}
typedef long long ll;
ll pow_mod(ll a, ll n) {
    ll res = 1;
    while(n) {
        if(n&1) res = res * a % MOD;
        a = a * a % MOD;
        n >>= 1;
    }
    return res;
}
\end{lstlisting}
\end{block}
\end{frame}
\fi

\end{document}